\begin{tcolorbox}[width=\textwidth,colback=gray!2!white,colframe=gray!50!blue]
    \textbf{Task:} \textit{Meta Prompting for In-Context Prompt Design}
    \begin{enumerate}
        \item \textbf{Document Analysis:} 
            \begin{itemize}
                \item Input: [Complex document, e.g., research paper, or even including this prompt itself]
                \item Action: Analyze and comprehend key concepts, methodologies, challenges, and objectives.
            \end{itemize}

        \item \textbf{Task Interpretation:} 
            \begin{itemize}
                \item Action: Synthesize information to define the core problem or task.
                \item Considerations: Identify constraints, goals, or requirements.
            \end{itemize}

        \item \textbf{Prompt Design:} 
            \begin{itemize}
                \item Objective: Develop a structured prompt for problem-solving.
                \item Elements: Instructions, step-by-step approach, background information.
            \end{itemize}

        \item \textbf{Optional - Direct Solution Proposal:}
            \begin{itemize}
                \item Objective: Propose initial steps or a complete solution strategy.
                \item Considerations: Feasibility and practicality within the context.
            \end{itemize}

        \item \textbf{Output Prompt: [to be generated using the same latex format as this prompt]}
    \end{enumerate}
    \textit{Note: The output is a coherent, actionable prompt or solution strategy, tailored to the specifics of the input document.}
\end{tcolorbox}
